\documentclass[10pt, a4paper, oneside]{ctexart}
% 导入包 %
\usepackage[
    paper=a4paper,
    left=1.5cm,
    right=1.5cm,
    top=1.5cm,
    bottom=1.5cm
]{geometry}
\usepackage{luacode} % 支持lua %
\usepackage{graphicx}  % 处理图片
\usepackage{float}
\usepackage[absolute,overlay]{textpos}  % 启用绝对定位
\usepackage{array}
\usepackage{url}

\begin{document}
\pagenumbering{gobble}
\setlength{\parindent}{0pt}

% lua %
\begin{luacode}
-- 使用lua读取,这样方便修改
local json = dofile("dkjson.lua")  -- 需要 lua-cjson 或 lualatex-platform 支持
local file = io.open("data.json", "r")
local content = file:read("*all")
file:close()
local data = json.decode(content)

-- 一些格式设置
local settings = data.settings
tex.print("\\gdef\\lineSize{" .. settings.horizonlineSize .. "}")
-- 是否渲染的设置
if settings.noUniversityBadge then
    tex.print("\\gdef\\noUniversityBadge{1}")
end
if settings.basicInfo then
    tex.print("\\gdef\\basicInfo{1}")
end
if settings.educationBackground then
    tex.print("\\gdef\\educationBackground{1}")
end
if settings.scientificResearch then
    tex.print("\\gdef\\scientificResearch{1}")
end
if settings.patent then
    tex.print("\\gdef\\patent{1}")
end
if settings.contest then
    tex.print("\\gdef\\contest{1}")
end
if settings.certificate then
    tex.print("\\gdef\\certificate{1}")
end
if settings.otherSkill then
    tex.print("\\gdef\\otherSkill{1}")
end
if settings.studentJob then
    tex.print("\\gdef\\studentJob{1}")
end
if settings.personalPhoto then
    tex.print("\\gdef\\personalPhoto{1}")
end
if settings.showCET4 then
    tex.print("\\gdef\\showCETiv{1}")
end
if settings.showCET6 then
    tex.print("\\gdef\\showCETvi{1}")
end
if settings.showEngLevel then
    tex.print("\\gdef\\showEngLevel{1}")
end
if settings.qualified then
    tex.print("\\gdef\\qualified{1}")
end
if settings.selfEvaluate then
    tex.print("\\gdef\\selfEvaluate{1}")
end

tex.print("\\gdef\\myname{" .. data.name .. "}") 
tex.print("\\gdef\\mytitle{" .. data.title .. "}")
tex.print("\\gdef\\target{" .. data.target .. "}")
-- 个人信息
personalInfo = data.personalInfo
tex.print("\\gdef\\figure{" .. personalInfo.figure .. "}")
tex.print("\\gdef\\tel{" .. personalInfo.tel .. "}")
tex.print("\\gdef\\gender{" .. personalInfo.gender .. "}")
tex.print("\\gdef\\birthday{" .. personalInfo.birthday .. "}")
tex.print("\\gdef\\tel{" .. personalInfo.tel .. "}")
tex.print("\\gdef\\nationality{" .. personalInfo.nationality .. "}")
tex.print("\\gdef\\ancestralHomeland{" .. personalInfo.ancestralHomeland .. "}")
tex.print("\\gdef\\politicStatue{" .. personalInfo.politicStatue .. "}")
-- 教育背景
local edu = data.educationBackground
tex.print("\\gdef\\UniversityBadge{" .. edu.badge .. "}")
tex.print("\\gdef\\university{" .. edu.university .. "}")
tex.print("\\gdef\\major{" .. edu.major .. "}")
tex.print("\\gdef\\level{" .. edu.level .. "}")
tex.print("\\gdef\\gpa{" .. edu.gpa .. "}")
tex.print("\\gdef\\grade{" .. edu.grade .. "}")
tex.print("\\gdef\\rank{" .. edu.rank .. "}")
-- 英语水平
local engLevel = data.english
if engLevel.cet4 ~= "" then
    tex.print("\\gdef\\cetIV{" .. engLevel.cet4 .. "}")
end
if engLevel.cet6 ~= "" then
    tex.print("\\gdef\\cetVI{" .. engLevel.cet6 .. "}")
end

\end{luacode}

% 校徽、姓名、联系方式、大头照 %
\begin{center}
    \begin{minipage}[t][6em]{1\textwidth}
        \ifdefined\noUniversityBadge
            \hspace{2em}
            {\Huge \textbf{\myname·\mytitle}}\\[1ex]
        \else
            % 姓名
            \begin{center}
                {\Huge \textbf{\myname}}\\[1ex]
            \end{center}
            % 校徽
            % 设置一个 10cm x 10cm 的图片,左边距 2cm,上边距 5cm
            \begin{textblock*}{3cm}(1cm,0.5cm)  % {宽度}(左边距, 上边距)
                \includegraphics[width=3cm]{\UniversityBadge}  % 图片宽度设为 50% 文档宽度
            \end{textblock*}
        \fi
        % 大头照
        \ifdefined\personalPhoto
        \begin{textblock*}{2.5cm}(16.5cm,0.5cm)  % {宽度}(左边距, 上边距)
            \includegraphics[width=2.5cm]{\figure}  % 图片宽度设为 50% 文档宽度
        \end{textblock*}
        \fi
        % 联系方式
        \vfil
        \begin{center}
            \begin{minipage}{1\textwidth}
                \centering
                \begin{minipage}{0.5\linewidth}
                    \textbf{Tel:} \tel
                \end{minipage}
                \hfill\begin{minipage}{0.5\linewidth}
                    \textbf{Email:}
                    \begin{luacode}
                        local json = dofile("dkjson.lua")  -- 需要 lua-cjson 或 lualatex-platform 支持
                        local file = io.open("data.json", "r")
                        local content = file:read("*all")
                        file:close()
                        local data = json.decode(content)
                        personalInfo = data.personalInfo
                        if personalInfo.email then
                            tex.print("\\url{" .. personalInfo.email .. "}")
                        end
                    \end{luacode}
                \end{minipage}
            \end{minipage}
        \end{center}
    \end{minipage}\\[1em]
\end{center}

\ifdefined\basicInfo
% 基本信息
\begin{minipage}{1\textwidth}
    \large{\textbf{基本信息}}
\end{minipage}
\rule{\linewidth}{\lineSize}
\begin{center}
    \begin{minipage}{0.8\textwidth}
        \begin{tabular}{@{}l p{3cm} @{\hskip 20pt} l l l l@{}}
        \textbf{性\hspace{2em}别:} &   \gender & \textbf{民族:} &   \nationality & \textbf{政治面貌:} &   \politicStatue \\
        \textbf{出生年月:}    &   \birthday & \textbf{籍贯:} &   \ancestralHomeland & \textbf{求学意向:} &   \target \\
    \end{tabular}
    \end{minipage}
\end{center}
\fi

\ifdefined\educationBackground
% 教育背景
\begin{minipage}{1\textwidth}
    \large{\textbf{教育背景}}
\end{minipage}
\rule{\linewidth}{\lineSize}
\begin{center}
    % 最外层,指定宽度
    \begin{minipage}{0.9\textwidth}
        % 学校和专业
        \begin{minipage}{0.6\linewidth}
            \university
            \hspace{2em}
            \textbf{\level}
            \hfil
        \end{minipage}
        \begin{minipage}{0.3\linewidth}
            \raggedright
            \textbf{专业}: \major
            \hfil
        \end{minipage} \\
        % GPA、加权、排名
        \begin{minipage}{0.2\linewidth}
            \textbf{GPA}: \gpa \hfil
        \end{minipage}
        \begin{minipage}{0.2\linewidth}
            \textbf{加权成绩}: \grade \hfil
        \end{minipage}
        \begin{minipage}{0.5\linewidth}
            \raggedright
            \textbf{排名}: \rank \ifdefined\qualified \hspace{2em} \textbf{已确认获取保研资格} \fi  \hfil
        \end{minipage} \\
        % 英语水平
        \ifdefined\showEngLevel
        \begin{minipage}{0.5\linewidth}
            \textbf{英语水平}:
            \hspace{1em}
            \ifdefined\showCETiv
            \textbf{CET-4} \ifdefined\cetIV : \cetIV \fi \hspace{2em}
            \fi
            \ifdefined\showCETvi
            \textbf{CET-6} \ifdefined\cetVI : \cetVI \fi \hfil
            \fi
        \end{minipage} \\[0.8em]
        \fi

        % 课程成绩
        \begin{minipage}{1\linewidth}
            \textbf{核心课程成绩}:
            \vspace{-2em}
            \begin{center}
                \begin{luacode}
                    -- 假设 JSON 数据
                    -- 使用lua读取,这样方便修改
                    local json = dofile("dkjson.lua")  -- 需要 lua-cjson 或 lualatex-platform 支持
                    local file = io.open("data.json", "r")
                    local content = file:read("*all")
                    file:close()
                    local data = json.decode(content)
                    local courses = data.educationBackground.courses
                    local cols = 3  -- 每行 3 列
                    local count = #courses
                    tex.print("\\begin{tabular}{lll}")  -- 定义 3 列表格
                    -- tex.print("\\hline")  -- 画横线
                    for i = 1, count, cols do
                        local row = {}
                        for j = 0, cols - 1 do
                            local index = i + j
                            if index <= count then
                                table.insert(row, courses[index][1] .. "(" .. courses[index][2] .. ")")
                            else
                                table.insert(row, "")  -- 补空白
                            end
                        end
                        tex.print(table.concat(row, " & ") .. " \\\\")  -- 拼接成 LaTeX 格式
                        -- tex.print("\\hline")
                    end
                    tex.print("\\end{tabular}")
                \end{luacode}
            \end{center}
        \end{minipage}
    \end{minipage}
\end{center}
\fi

\ifdefined\scientificResearch
% 科研经历
\begin{minipage}{1\textwidth}
    \large{\textbf{科研经历}}
\end{minipage}
\rule{\linewidth}{\lineSize}
\begin{center}
    \begin{minipage}{0.9\textwidth}
        \begin{luacode}
            local json = dofile("dkjson.lua")  -- 需要 lua-cjson 或 lualatex-platform 支持
            local file = io.open("data.json", "r")
            local content = file:read("*all")
            file:close()
            local data = json.decode(content)
            local sR = data.scientificResearch
            for _ , i in ipairs(sR) do
                tex.print("\\begin{minipage}{1\\textwidth}")

                -- 项目名
                tex.print("\\begin{minipage}{0.5\\linewidth}")
                tex.print("\\raggedright")
                tex.print("\\textbf{" .. i.project .. "}")
                -- tex.print("\\hfil")
                tex.print("\\end{minipage}")
                -- 备注
                tex.print("\\begin{minipage}{0.2\\linewidth}")
                tex.print("\\centering")
                tex.print(i.remark)
                -- tex.print("\\hfil")
                tex.print("\\end{minipage}")
                -- 时间
                tex.print("\\begin{minipage}{0.2\\linewidth}")
                tex.print("\\raggedright")
                tex.print(i.time)
                -- tex.print("\\hfil")
                tex.print("\\end{minipage}")

                tex.print("\\\\[0.8em]")

                -- 使用列表代替,方便写课程设计等
                for _ , detail in ipairs(i.detail) do
                    tex.print("\\begin{minipage}{1\\linewidth}")
                    tex.print("\\raggedright")
                    tex.print("- " .. detail[1] .. ":" .. detail[2])
                    tex.print("\\hfil")
                    tex.print("\\end{minipage}")
                end
                tex.print("\\end{minipage}")
                tex.print("\\\\[1em]")
            end
        \end{luacode}
    \end{minipage}
\end{center}
\fi

\ifdefined\patent
% 专利
\begin{minipage}{1\textwidth}
    \large{\textbf{科研创新}}
\end{minipage}
\rule{\linewidth}{\lineSize}
\begin{center}
    \begin{minipage}{0.9\textwidth}
        \begin{luacode}
             local json = dofile("dkjson.lua")  -- 需要 lua-cjson 或 lualatex-platform 支持
            local file = io.open("data.json", "r")
            local content = file:read("*all")
            file:close()
            local data = json.decode(content)
            local patents = data.patent
            for _ , patent in ipairs(patents) do
                tex.print("\\begin{minipage}{1\\linewidth}")

                -- 专利名称
                tex.print("\\begin{minipage}{0.6\\linewidth}")
                tex.print("\\raggedright")
                tex.print("- " .. patent[1])
                tex.print("\\hfil")
                tex.print("\\end{minipage}")

                -- 专利号
                tex.print("\\begin{minipage}{0.3\\linewidth}")
                tex.print("\\raggedright")
                tex.print(patent[2])
                tex.print("\\hfil")
                tex.print("\\end{minipage}")

                tex.print("\\end{minipage}\\\\[0.5em]")
            end
        \end{luacode}
    \end{minipage}
\end{center}
\fi

\ifdefined\contest
% 竞赛经历
\begin{minipage}{1\textwidth}
    \large{\textbf{竞赛经历}}
\end{minipage}
\rule{\linewidth}{\lineSize}
\begin{center}
    \begin{minipage}{0.9\textwidth}
        \begin{luacode}
            local json = dofile("dkjson.lua")  -- 需要 lua-cjson 或 lualatex-platform 支持
            local file = io.open("data.json", "r")
            local content = file:read("*all")
            file:close()
            local data = json.decode(content)
            local contests = data.contest
            for _ , contest in ipairs(contests) do
                tex.print("\\begin{minipage}{1\\linewidth}")

                -- 竞赛时间及名称
                tex.print("\\begin{minipage}{0.8\\linewidth}")
                tex.print(" - " .. contest[1])
                tex.print("\\hfil")
                tex.print("\\end{minipage}")

                -- 竞赛等级
                tex.print("\\begin{minipage}{0.1\\linewidth}")
                tex.print(contest[2])
                tex.print("\\end{minipage}")

                tex.print("\\end{minipage}\\\\[0.5em]")
            end
        \end{luacode}
    \end{minipage}
\end{center}
\fi

\ifdefined\certificate
% 荣誉奖项
\begin{minipage}{1\textwidth}
    \large{\textbf{荣誉奖项}}
\end{minipage}
\rule{\linewidth}{\lineSize}
\begin{center}
    \begin{minipage}{0.9\textwidth}
        \begin{luacode}
            local json = dofile("dkjson.lua")  -- 需要 lua-cjson 或 lualatex-platform 支持
            local file = io.open("data.json", "r")
            local content = file:read("*all")
            file:close()
            local data = json.decode(content)
            local certificates = data.certificate
            for _ , certificate in ipairs(certificates) do
                tex.print("\\begin{minipage}{1\\linewidth}")

                -- 获奖名
                tex.print("\\begin{minipage}{0.8\\linewidth}")
                tex.print(" - " .. certificate[1])
                tex.print("\\hfil")
                tex.print("\\end{minipage}")

                -- 获奖时间
                tex.print("\\begin{minipage}{0.1\\linewidth}")
                tex.print(certificate[2])
                tex.print("\\end{minipage}")

                tex.print("\\end{minipage}\\\\[0.5em]")
            end
        \end{luacode}
    \end{minipage}
\end{center}
\fi

\ifdefined\otherSkill
% 基本技能
\begin{minipage}{1\textwidth}
    \large{\textbf{基本技能}}
\end{minipage}
\rule{\linewidth}{\lineSize}
\begin{center}
    \begin{minipage}{0.9\textwidth}
        \begin{luacode}
            local json = dofile("dkjson.lua")  -- 需要 lua-cjson 或 lualatex-platform 支持
            local file = io.open("data.json", "r")
            local content = file:read("*all")
            file:close()
            local data = json.decode(content)
            local skills = data.otherSkill
            for _ , skill in ipairs(skills) do
                tex.print("\\begin{minipage}{1\\linewidth}")
                tex.print(" - " .. skill[1] .. ":" .. skill[2])
                tex.print("\\end{minipage}\\\\[0.5em]")
            end
        \end{luacode}
    \end{minipage}
\end{center}
\fi

\ifdefined\studentJob
% 学生工作
\begin{minipage}{1\textwidth}
    \large{\textbf{学生工作}}
\end{minipage}
\rule{\linewidth}{\lineSize}
\begin{center}
    \begin{minipage}{0.9\textwidth}
        \begin{luacode}
            local json = dofile("dkjson.lua")  -- 需要 lua-cjson 或 lualatex-platform 支持
            local file = io.open("data.json", "r")
            local content = file:read("*all")
            file:close()
            local data = json.decode(content)
            local jobs = data.studentJob
            for _ , job in ipairs(jobs) do
                tex.print("\\begin{minipage}{1\\linewidth}")

                -- 时间段
                tex.print("\\begin{minipage}{0.4\\linewidth}")
                tex.print("- " .. job[1])
                tex.print("\\hfil")
                tex.print("\\end{minipage}")

                -- 内容
                tex.print("\\begin{minipage}{0.5\\linewidth}")
                tex.print("\\raggedright")
                tex.print(job[2])
                tex.print("\\end{minipage}")

                tex.print("\\end{minipage}\\\\[0.5em]")
            end
        \end{luacode}
    \end{minipage}
\end{center}
\fi

\ifdefined\selfEvaluate
% 自我评价
\begin{minipage}{1\textwidth}
    \large{\textbf{自我评价}}
\end{minipage}
\rule{\linewidth}{\lineSize}
\begin{center}
    \begin{minipage}{0.9\textwidth}
        \begin{luacode}
            local json = dofile("dkjson.lua")  -- 需要 lua-cjson 或 lualatex-platform 支持
            local file = io.open("data.json", "r")
            local content = file:read("*all")
            file:close()
            local data = json.decode(content)
            local evaluate = data.selfEvaluate
            for _ , text in ipairs(evaluate) do
                tex.print("\\begin{minipage}{1\\linewidth}")

                tex.print(" - " .. text)

                tex.print("\\end{minipage}\\\\[0.5em]")
            end
        \end{luacode}
    \end{minipage}
\end{center}
\fi

\end{document}